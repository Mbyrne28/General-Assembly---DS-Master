\documentclass[12pt,aspectratio=169]{beamer}

\usetheme{metropolis}

\definecolor{mDarkBrown}{HTML}{FF5722}
\definecolor{mDarkTeal}{HTML}{263238}
\definecolor{mLightBrown}{HTML}{FF5722}

\usepackage{booktabs}
\usepackage{graphicx}
\usepackage{hyphenat}
\usepackage{multirow}
\usepackage{nicefrac}
\usepackage[normalem]{ulem}

\usepackage[weather]{ifsym}

\usepackage{pifont}
\newcommand{\cmark}{\ding{51}}
\newcommand{\xmark}{\ding{55}}

\usepackage{minted}
\usemintedstyle{tango}
\newminted[bash]{bash}{%
    autogobble,
    bgcolor=mDarkTeal!10,
    linenos
}
\newminted[py3]{python}{%
    python3,
    autogobble,
    bgcolor=mDarkTeal!10,
    linenos
}
\newminted[sql]{sql}{%
    autogobble,
    bgcolor=mDarkTeal!10,
    linenos
}

\usepackage{polyglossia}
\setdefaultlanguage[variant=british]{english}
\usepackage[english=british]{csquotes}

\defaultfontfeatures{Ligatures=TeX}
\setmainfont{Lucida Sans OT}
\setsansfont[Scale=MatchLowercase]{Lucida Sans OT}
\setmonofont[Scale=MatchLowercase]{Lucida Console DK}

\usepackage{mathspec}
\setmathsfont(Digits,Latin,Greek)[Numbers={Lining,Proportional}]{Lucida Bright Math OT}

\newcommand{\mat}[1]{\ensuremath{\mathbf{#1}}}

\newcommand{\R}{\ensuremath{\mathbb{R}}}

\newcommand{\E}[1]{\ensuremath{\mathbb{E}\!\left[ #1 \right]}}
\newcommand{\V}[1]{\ensuremath{\mathbb{V}\!\left[ #1 \right]}}
\newcommand{\Prob}[1]{\ensuremath{\Pr\!\left( #1 \right)}}
\newcommand{\Normal}[2]{\ensuremath{\mathcal{N}\!\left( #1, #2 \right)}}
\newcommand{\simiid}{\ensuremath{\overset{\text{\tiny i.i.d.}}{\sim}}}

\DeclareMathOperator{\logit}{logit}

\author{Gianluca Campanella}
\date{}



\title{Introduction to databases}

\begin{document}

\maketitle

\begin{frame}{Contents}
    \tableofcontents[hideallsubsections]
\end{frame}

\begin{frame}{Databases}
    Databases manage\ldots
    \begin{itemize}
        \item \alert{Storage} of information \\[\bigskipamount]
        \item \alert{Querying} and retrieval
        \item[$\rightarrow$] Structured Query Language (SQL) \\[\bigskipamount]
        \item \alert{Consistency} and \alert{access rights} (permissions)
    \end{itemize}
\end{frame}

\section{Relational databases}

\begin{frame}{Relational databases}
    \begin{itemize}
        \item Organised in \alert{tables} (‘entities’ or ‘concepts’) \\[\bigskipamount]
        \item Each table has a \alert{schema} describing data types and constraints \\[\bigskipamount]
        \item In addition, tables have \alert{keys} that serve as\ldots
              \begin{itemize}
                  \item Identifiers (\alert{primary key})
                  \item Indices (\alert{secondary keys})
              \end{itemize}
    \end{itemize}
\end{frame}

\begin{frame}{Database management systems (DBMS)}
    \begin{block}{Open\hyp{}source}
        \begin{itemize}
            \item MySQL and derivatives
            \item PostgreSQL
        \end{itemize}
    \end{block}
    \vfill
    \begin{block}{Commercial}
        \begin{itemize}
            \item Microsoft SQL Server
            \item Oracle
        \end{itemize}
    \end{block}
\end{frame}

\begin{frame}{Normalisation}
    A normalised database has:\vspace{-1ex}
    \begin{itemize}
        \item One table per entity
        \item Many foreign keys and/or associative tables
    \end{itemize}
    \vfill\pause
    \begin{columns}[t]
        \begin{column}{0.5\textwidth}
            \begin{center}
                \textbf{Pros}
            \end{center}
            \vspace{-1ex}
            \begin{itemize}
                \item Minimal data duplication
                \item Saves storage space
            \end{itemize}
        \end{column}
        \begin{column}{0.5\textwidth}
            \begin{center}
                \textbf{Cons}
            \end{center}
            \vspace{-1ex}
            \begin{itemize}
                \item Split across different tables
                \item Requires joins to ‘reconstruct’
            \end{itemize}
        \end{column}
    \end{columns}
\end{frame}

\section{Other database types}

\begin{frame}{Key-value stores}
    A key\hyp{}value store\ldots\vspace{-1ex}
    \begin{itemize}
        \item Is like a Python \alert{dictionary}, but not limited to available
              memory
        \item Uses \alert{caching} strategies to ensure quick access to
              commonly or recently accessed items
    \end{itemize}
    \vfill
    \begin{block}{Examples}
        \begin{itemize}
            \item Apache Cassandra
            \item Oracle NoSQL Database
        \end{itemize}
    \end{block}
\end{frame}

\begin{frame}{NoSQL databases}
    A NoSQL database\ldots\vspace{-1ex}
    \begin{itemize}
        \item Organises data in ‘entities’ that allow for \alert{nesting}
        \item Typically describes data using \alert{JSON}
    \end{itemize}
    \vfill
    \begin{block}{Examples}
        \begin{itemize}
            \item Apache CouchDB
            \item MongoDB
        \end{itemize}
    \end{block}
\end{frame}

\section{Structured Query Language}

\begin{frame}[t,fragile]{Selecting data}
    \begin{onlyenv}<1>
        \begin{block}{Syntax}
            \begin{sql}
                SELECT <columns>
                FROM <table>
                WHERE <conditions>
            \end{sql}
        \end{block}
        \vfill
        \begin{block}{Notes}
            \begin{itemize}
                \item \mintinline{sql}{SELECT *} will select \alert{all columns}
                \item \mintinline{sql}{WHERE} can be omitted to retrieve
                      \alert{all rows}
            \end{itemize}
        \end{block}
    \end{onlyenv}
    \begin{onlyenv}<2>
        \begin{block}{Example}
            \begin{sql}
                SELECT store, sales
                FROM global_sales
                WHERE country == 'UK'
            \end{sql}
        \end{block}
    \end{onlyenv}
\end{frame}

\begin{frame}[t,fragile]{Grouping}
    \begin{onlyenv}<1>
        \begin{block}{Syntax}
            \begin{sql}
                SELECT STATISTIC(<column>), ...
                FROM <table>
                ...
                GROUP BY <indices>
            \end{sql}
        \end{block}
        \vfill
        \begin{block}{Notes}
            \begin{itemize}
                \item Usually paired with a \mintinline{sql}{STATISTIC} such as
                      \mintinline{sql}{COUNT}, \mintinline{sql}{SUM},
                      \mintinline{sql}{AVG}, \mintinline{sql}{MIN},
                      or \mintinline{sql}{MAX} computed within groups
                \item \mintinline{sql}{GROUP BY} can be omitted to aggregate
                      over \alert{all rows}
            \end{itemize}
        \end{block}
    \end{onlyenv}
    \begin{onlyenv}<2>
        \begin{block}{Example}
            \begin{sql}
                SELECT store, SUM(sales)
                FROM global_sales
                WHERE country == 'UK'
                GROUP BY store
            \end{sql}
        \end{block}
    \end{onlyenv}
\end{frame}

\begin{frame}[t,fragile]{Ordering}
    \begin{onlyenv}<1>
        \begin{block}{Syntax}
            \begin{sql}
                SELECT <columns>
                FROM <table>
                ...
                ORDER BY <indices> [DESC]
            \end{sql}
        \end{block}
        \vfill
        \begin{block}{Notes}
            \begin{itemize}
                \item Default sorting is in \mintinline{sql}{ASC}ending order
                \item Can also \mintinline{sql}{ORDER BY} multiple columns
            \end{itemize}
        \end{block}
    \end{onlyenv}
    \begin{onlyenv}<2>
        \begin{block}{Example 1}
            \begin{sql}
                SELECT country, city, store
                FROM global_sales
                ORDER BY country, city
            \end{sql}
        \end{block}
        \vfill
        \begin{block}{Example 2}
            \begin{sql}
                SELECT store, SUM(sales) AS total_sales
                FROM global_sales
                GROUP BY store
                ORDER BY total_sales DESC
            \end{sql}
        \end{block}
    \end{onlyenv}
\end{frame}

\begin{frame}[t,fragile]{Joining}
    \begin{onlyenv}<1>
        \begin{block}{Syntax}
            \begin{sql}
                SELECT <columns>
                FROM <table>
                JOIN <table> ON <conditions>
                ...
            \end{sql}
        \end{block}
        \vfill
        \begin{block}{Notes}
            \begin{itemize}
                \item Performs an \alert{inner join} (matching \alert{both}
                      tables)
                \item \alert{Outer joins} (\mintinline{sql}{LEFT},
                      \mintinline{sql}{RIGHT}, or \mintinline{sql}{FULL}) can
                      also be performed
            \end{itemize}
        \end{block}
    \end{onlyenv}
    \begin{onlyenv}<2>
        \begin{block}{Example}
            \begin{sql}
                SELECT st.city, st.store, SUM(sa.sales) AS total_sales
                FROM stores AS st
                JOIN global_sales AS sa ON st.store == sa.store
                WHERE st.country == 'UK'
                GROUP BY st.country
                ORDER BY total_sales DESC
            \end{sql}
        \end{block}
    \end{onlyenv}
\end{frame}

\end{document}

