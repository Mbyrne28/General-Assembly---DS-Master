\documentclass[12pt,aspectratio=169]{beamer}

\usetheme{metropolis}

\definecolor{mDarkBrown}{HTML}{FF5722}
\definecolor{mDarkTeal}{HTML}{263238}
\definecolor{mLightBrown}{HTML}{FF5722}

\usepackage{booktabs}
\usepackage{graphicx}
\usepackage{hyphenat}
\usepackage{multirow}
\usepackage{nicefrac}
\usepackage[normalem]{ulem}

\usepackage[weather]{ifsym}

\usepackage{pifont}
\newcommand{\cmark}{\ding{51}}
\newcommand{\xmark}{\ding{55}}

\usepackage{minted}
\usemintedstyle{tango}
\newminted[bash]{bash}{%
    autogobble,
    bgcolor=mDarkTeal!10,
    linenos
}
\newminted[py3]{python}{%
    python3,
    autogobble,
    bgcolor=mDarkTeal!10,
    linenos
}
\newminted[sql]{sql}{%
    autogobble,
    bgcolor=mDarkTeal!10,
    linenos
}

\usepackage{polyglossia}
\setdefaultlanguage[variant=british]{english}
\usepackage[english=british]{csquotes}

\defaultfontfeatures{Ligatures=TeX}
\setmainfont{Lucida Sans OT}
\setsansfont[Scale=MatchLowercase]{Lucida Sans OT}
\setmonofont[Scale=MatchLowercase]{Lucida Console DK}

\usepackage{mathspec}
\setmathsfont(Digits,Latin,Greek)[Numbers={Lining,Proportional}]{Lucida Bright Math OT}

\newcommand{\mat}[1]{\ensuremath{\mathbf{#1}}}

\newcommand{\R}{\ensuremath{\mathbb{R}}}

\newcommand{\E}[1]{\ensuremath{\mathbb{E}\!\left[ #1 \right]}}
\newcommand{\V}[1]{\ensuremath{\mathbb{V}\!\left[ #1 \right]}}
\newcommand{\Prob}[1]{\ensuremath{\Pr\!\left( #1 \right)}}
\newcommand{\Normal}[2]{\ensuremath{\mathcal{N}\!\left( #1, #2 \right)}}
\newcommand{\simiid}{\ensuremath{\overset{\text{\tiny i.i.d.}}{\sim}}}

\DeclareMathOperator{\logit}{logit}

\author{Gianluca Campanella}
\date{}



\title{Decision trees and random forests}

\begin{document}

\maketitle

\begin{frame}{Contents}
    \tableofcontents[hideallsubsections]
\end{frame}

\section{Decision trees}

\begin{frame}{Should we wait?}
    \only<1>{%
        \begin{block}{Problem}
            You're out with friends and need to decide whether to wait for a
            table at a busy restaurant.
            You have the following information:
            \begin{itemize}
                \small
                \item Whether there's an alternative restaurant nearby
                \item Whether the restaurant has a bar
                \item How busy the restaurant is (empty, some people, packed)
                \item Whether you're hungry (not at all, peckish, starving)
                \item Whether it's raining
                \item Type of restaurant (British, Chinese, Italian or Thai)
                \item Whether it's Friday or Saturday night
            \end{itemize}
        \end{block}}
    \only<2>{%
        \begin{block}{Idea}
            Imagine taking a sequence of decisions:
            \begin{itemize}
                \item If the restaurant is packed\ldots
                      \begin{itemize}
                          \item \ldots and we're starving\ldots
                          \begin{itemize}
                              \item \ldots but there's no alternative in the
                                     area $\rightarrow$ wait
                          \end{itemize}
                      \end{itemize}
            \end{itemize}
        \end{block}
        \vfill
        \begin{block}{Question}
            How `tall' should the decision tree grow?
        \end{block}}
\end{frame}

\begin{frame}{Expressiveness}
    \begin{itemize}
        \item Decision trees can express \alert{any} function of the predictors \\
              (using one leaf per sample) \\[\medskipamount]
        \item We want to find \alert{structure} in the data, not overfit
        \item[$\rightarrow$] Compact trees
    \end{itemize}
    \vfill\pause
    \begin{block}{Idea}
        \begin{itemize}
            \item Choose `most significant' attribute as (sub)root
            \item[$\rightarrow$] Ideally achieving perfect separation of categories \\[\medskipamount]
            \item Repeat (recursively)
        \end{itemize}
    \end{block}
\end{frame}

\begin{frame}{Comparison with logistic regression}
    \begin{itemize}
        \item Logistic regression is a linear model
        \item Each predictor `acts' independently of all others \\
              (its marginal effect is the regression coefficient)
    \end{itemize}
    \vfill\pause
    This doesn't always work:
    \vspace{-0.5em}
    \begin{itemize}
        \item If the restaurant is packed\ldots
              \begin{itemize}
                  \item \ldots and we're starving\ldots
                  \begin{itemize}
                      \item \ldots but \alert{there is} an alternative
                            $\rightarrow$ do we still wait?
                  \end{itemize}
              \end{itemize}
    \end{itemize}
    Decision trees automatically contain \alert{interactions}, since each
    `question' depends on the previous one
\end{frame}

\begin{frame}{Training}
    \begin{itemize}
        \item Start with the entire dataset \\[\bigskipamount]
        \item Find the `question' that best segregates the samples
              $\rightarrow$ \alert{purity} \\[\bigskipamount]
        \item Repeat (recursively) until:
              \begin{itemize}
                  \item You have asked as many questions as you wanted
                  \item The gain in purity of possible splits is negligible
                  \item Leaves are completely pure
              \end{itemize}
    \end{itemize}
\end{frame}

\begin{frame}{Prediction}
    \begin{itemize}
        \item Answer each `question' until you reach a leaf \\[\bigskipamount]
        \item Take the majority label of samples in that leaf
    \end{itemize}
\end{frame}

\begin{frame}{Purity metrics}
    \begin{block}{Gini impurity}
        \begin{itemize}
            \item How often would a randomly chosen sample be labelled
                  incorrectly if it was labelled randomly with class proportions
                  $p_{i}$?
            \item[$\rightarrow$] $\sum_{k} p_{k} \left( 1 - p_{k} \right)$
        \end{itemize}
    \end{block}
    \vfill
    \begin{block}{Information gain}
        \begin{itemize}
            \item What's the reduction in (Shannon) entropy?
            \item[$\rightarrow$] Difference in $- p \log\!\left( p \right)$ from
                                 parent to children
        \end{itemize}
    \end{block}
\end{frame}

\begin{frame}{Overfitting}
    \begin{itemize}
        \item Decision trees can easily `memorise' the data
        \item[$\rightarrow$] Overfitting
    \end{itemize}
    \vfill\pause
    \begin{block}{Solutions}
        Impose a limit on the\ldots\vspace{-1ex}
        \begin{itemize}
            \item Maximum number of questions (depth)
            \item Minimum number of samples in each leaf
        \end{itemize}
    \end{block}
\end{frame}

\begin{frame}{Pros and cons}
    \begin{block}{Pros}
        \begin{itemize}
            \item Can be used for regression or classification
            \item Can be visualised $\rightarrow$ easy to interpret
            \item Correspond to a series of `rules'
            \item Learn interactions and irrelevant predictors
        \end{itemize}
    \end{block}
    \vfill
    \begin{block}{Cons}
        \begin{itemize}
            \item Prone to overfitting and sensitive to small variations
            \item May not be globally optimal because of `greedy' splitting
            \item Don't work well with unbalanced classes or small datasets
        \end{itemize}
    \end{block}
\end{frame}

\section{Random forests}

\begin{frame}{Bagging}
    Imagine a situation where\ldots
    \begin{itemize}
        \item You have many different models
        \item Each predicts your outcome with some accuracy
        \item Each also makes (independent) errors
    \end{itemize}
    How could you improve your prediction?
    \vfill\pause
    \begin{block}{Idea}
        Let all classifiers predict and take the majority vote \\
        (or the mean for continuous outcomes)
    \end{block}
\end{frame}

\begin{frame}{Random forests}
    \begin{itemize}
        \item A collection (\alert{ensemble}) of decision trees \\[\bigskipamount]
        \item Built randomly\ldots
              \begin{itemize}
                  \item On a subset of the data
                  \item Using a subset of predictors
              \end{itemize}
              \ldots to avoid overfitting \\[\bigskipamount]
        \item For prediction, each tree contributes an answer, and the final
              model prediction is the majority vote (or the mean for continuous
              outcomes)
    \end{itemize}
\end{frame}

\begin{frame}{Boosting}
    Imagine a situation where\ldots
    \begin{itemize}
        \item You are training many models of the same type
              \alert{sequentially}
        \item Each predicts your outcome\ldots
              \begin{itemize}
                  \item Correctly for some samples
                  \item Incorrectly for some other samples
              \end{itemize}
    \end{itemize}
    How could you improve your prediction?
    \vfill\pause
    \begin{block}{Idea}
        \begin{itemize}
            \item At each step, `refine' the model by giving more weight to
                  incorrectly predicted samples (the `hard' ones)
            \item For prediction, compute a weighted vote/mean of \alert{all}
                  predictions
        \end{itemize}
    \end{block}
\end{frame}

\end{document}

